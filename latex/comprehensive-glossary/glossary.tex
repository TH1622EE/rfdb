%% PLEASE ADD YOUR ENTRIES IN ALPHABETICAL ORDER WITH RESPECT TO THE OTHER
%% ENTRIES WHEN ADDING ADDITIONAL TERMS SO THEY ARE EASIER FOR OTHERS TO FIND,
%% COPY AND PASTE INTO THEIR OWN DOCUMENTS DURING CREATION.

%%%%%%%%%%%%%%%%%%%%%%%%%%%%%%%%%%%%%
% Beginning of the Glossary of Terms
%%%%%%%%%%%%%%%%%%%%%%%%%%%%%%%%%%%%%

\newglossaryentry{-3dB bandwidth}
{
    name = -3dB bandwidth,
    description = {the -3dB bandwidth (also known as the half-power bandwidth) is the frequency range over which the power of a component or systems response is at least 50\% of its peak value.}
}

\newglossaryentry{absolute permittivity}
{
    name = absolute permittivity,
    description = {a simplified real-valued material property used as a measure of a material's ability to store electric potential energy in the presence of a static electric field where the dielectric material is assumed to behave as a perfect insulator}
}
\newglossaryentry{axial ratio}
{
    name={axial ratio},
    description={The ratio of the major to minor axes of the polarization ellipse}
}
\newglossaryentry{bandwidth}
{
    name = bandwidth,
    description = {the range of frequencies within which performance, with respect to some characteristic, falls within specific limits}
}
\newglossaryentry{circular polarization}
{
    name={circular polarization},
    description={Polarization where the electric field vector has constant magnitude and rotates in a circle at the wave frequency}
}
\newglossaryentry{complex permittivity}
{
    name = complex permittivity,
    description = {a complex-valued frequency-dependent material property composed of a real and imaginary component used as a measure of a material's ability to store electric potential energy in the presence of an oscillating electric field}
}
\newglossaryentry{conductivity}{
    name={conductivity},
    description={Material property that describes its ability to conduct electric current, measured in S/m}
}
\newglossaryentry{dipole_antenna}
{
    name={dipole antenna},
    description={A simple antenna consisting of two conductive elements, often a half-wavelength in total length}
}
\newglossaryentry{directivity}
{
    name = directivity,
    description = {the ratio of the radiation intensity in a given direction from the antenna to the radiation intensity averaged over all directions}
}
\newglossaryentry{electric_field}
{
    name={electric field},
    description={Vector field representing the force per unit charge in space}
}
\newglossaryentry{elliptical_polarization}
{
    name={elliptical polarization},
    description={Polarization where the electric field vector describes an ellipse over time}
}
\newglossaryentry{frequency}
{
    name = frequency,
    description = {the number of events or cycles per second measured in units of hertz (Hz); the rate of a repetitive event}
}

\newglossaryentry{first-null beamwidth}
{
    name = first-null beamwidth,
    description = {the first-null bandwidth (FNBW) is the first frequency range (angular width) over which the power of a component or systems response is equal to zero}
}

\newglossaryentry{half-power beamwidth}
{
    name = half-power beamwidth,
    description = {the half-power beamwidth (also known as the -3dB beamwidth) is the frequency range (angular width) over which the power of a component or systems response is equal to 50\% of its peak power (i.e., -3 dB)}
}
\newglossaryentry{incident wave}
{
    name={incident wave},
    description={An electromagnetic wave impinging upon an antenna or boundary}
}
\newglossaryentry{infinitesimal linear_dipole}{
    name={infinitesimal linear dipole},
    description={A very short current element used as a fundamental antenna model, often $\ll \lambda$}
}
\newglossaryentry{input reactance}
{
    name={input reactance},
    description={Imaginary part of the input impedance of an antenna, representing stored energy}
}
\newglossaryentry{intrinsic impedance}
{
    name = intrinsic impedance,
    description = {a complex-valued fundamental property of a medium that characterizes its impedance to the propagation of electromagnetic energy defined as the ratio of the magnitudes of the electric field ($\Vec{E}$ to the magnetic field ($\Vec{H}$) of an \textit{Electromagnetic} (\acrshort{EM}) wave propagating through it as a function of the material's permittivity, permeability, and conductivity.}
}

\newglossaryentry{intrinsic impedance of free space}
{
    name = intrinsic impedance of free space,
    description = {a real-valued fundamental property of free space as a medium defined as the ratio of the square root of the ratio of vacuum permeability to vacuum permittivity of an electromagnetic wave propagating through a vacuum}
}
\newglossaryentry{linear polarization}
{
    name={linear polarization},
    description={Polarization where the electric field vector maintains a constant direction}
}
\newglossaryentry{loss resistance}
{
    name={loss resistance},
    description={Equivalent resistance representing conductor or dielectric losses in an antenna}
}
\newglossaryentry{magnetic field}
{
    name={magnetic field},
    description={Vector field representing the magnetic influence of electric currents and changing electric fields}
}
\newglossaryentry{narrowband}
{
    name = narrowband,
    description = {a signal or channel utilizing a bandwidth of less than 1-octave of the frequency of operation}
}
\newglossaryentry{period}
{
    name = period,
    description = {the duration of one complete cycle of a signal}
}
\newglossaryentry{polarization}
{
    name={polarization},
    description={Orientation and shape of the electric field vector as a function of time}
}
\newglossaryentry{polarization loss factor}
{
    name={polarization loss factor},
    description={Measure of mismatch between the polarization of an incoming wave and an antenna, defined as the squared magnitude of the polarization vector dot product}
}
\newglossaryentry{radiation efficiency}{
    name={radiation efficiency},
    description={Ratio of radiated power to total input power of an antenna}
}
\newglossaryentry{radiation resistance}{
    name={radiation resistance},
    description={Equivalent resistance that accounts for power radiated by the antenna}
}
\newglossaryentry{standing wave ratio}
{
    name = Standing Wave Ratio,
    description = {the ratio of forward propagating to reflected waves along a transmission line}
}
\newglossaryentry{tilt angle}
{
    name={tilt angle},
    description={angle between the major axis of the polarization ellipse and the reference axis}
}
\newglossaryentry{uniform plane wave}
{
    name={uniform plane wave},
    description={an electromagnetic wave with uniform amplitude and phase in planes normal to the direction of propagation}
}
\newglossaryentry{vswr}
{
    name={voltage standing wave ratio},
    description={ratio of the maximum to minimum voltage along a transmission line, representing impedance matching quality}
}
\newglossaryentry{wideband}
{
    name = wideband,
    description = {a signal or channel utilizing a bandwidth of more than 1-octave of the frequency of operation}
}

%%%%%%%%%%%%%%%%%%%%%%%%%%%%%%%%%%%%%
% Acronyms
%%%%%%%%%%%%%%%%%%%%%%%%%%%%%%%%%%%%%

\newacronym{ac}{AC}{Alternating Current}
\newacronym{ar}{AR}{Axial Ratio}
\newacronym{bpm}{bpm}{beats-per-minute}
\newacronym{cp}{CP}{Circular Polarization}
\newacronym{em}{EM}{electromagnetic}
\newacronym{ep}{EP}{Elliptical Polarization}
\newacronym{fnbw}{FNBW}{First-Null Beamwidth}
\newacronym{hpbw}{HPBW}{Half-Power Beamwidth}
\newacronym{lp}{LP}{Linear Polarization}
\newacronym{nb}{NB}{narrowband}
\newacronym{plf}{PLF}{Polarization Loss Factor}
\newacronym{rf}{RF}{Radio Frequency}
\newacronym{swr}{SWR}{Standing Wave Ratio}
\newacronym{ta}{TA}{Tilt Angle}
\newacronym{vswr}{VSWR}{Voltage Standing Wave Ratio}
\newacronym{wb}{WB}{wideband}
